
\documentclass{report}   

\usepackage{graphicx}
\usepackage[labelformat=empty]{caption}
\usepackage{subcaption}

\usepackage{multirow}
\usepackage{graphics,graphicx} % for pdf, bitmapped graphics files
\usepackage{epsfig} % for postscript graphics files
%\usepackage{mathptmx} % assume new font selection scheme installed
\usepackage{times} % assumes new font selection scheme installed
%\usepackage{amssymb}  % assumes amsmath package installed

\usepackage{amsmath} % assumes amsmath package installed
\usepackage{amsfonts}
%\usepackage{amssymb}  % assumes amsmath package installed
\usepackage{url}
\usepackage{hyperref} 

\newcommand{\Rnum}{\mathbb{R}} % Symbol fo the real numbers set
\newcommand{\mat}[1]{\ensuremath{\begin{bmatrix}#1\end{bmatrix}}}	% matrix
\newcommand{\myparagraph}[1]{\paragraph{#1}\mbox{}\\}

\begin{document}


\section*{LAB 2: Operational Space Control	}

\quad

\noindent
1) \textit{Generate Sinusoidal Reference:}
Generate sinuoidal reference for $X$ direction of the end effector with 0.1 amplitude and 1.5 Hz frequency.

\quad

\noindent
2) \textit{Step Reference generation:}
Generate a 0.1 step reference step(t = 2.0) for Z direction.

\quad

\noindent
3) \textit{Inverse kinematics:} (OPTIONAL) Implement the end-effector tracking with inverse kinematics using PD control (advice: $q^d$ = IK($x^d$), $\dot{q}^d =  J^{-1} \dot{x}^d$). Algorithm for IK: \href{https://gepettoweb.laas.fr/doc/stack-of-tasks/pinocchio/master/doxygen-html/md_doc_b-examples_i-inverse-kinematics.html}{pinocchio example}). Evaluate the tracking of the end-effector.

\quad

\noindent
4)\textit{ End-effector position  PD control:}
Implement End-effector PD control (simplified impedance control) with $K_x$ = 1000 N/m and $D_x$ = 300 Ns/m (advice consider only linear Jacobian 6x3). 
See what happens if you do not add any null-space torques. Then add them as a postural task (i.e. to attract to a default posture in the joint space), to remove the indeterminacy of the task that is only 3D.


\begin{align}
  \tau_0& = 50(q_0-q) - 10\dot{q}\\
\tau_N& = N \tau_0 
\label{fig:}
\end{align}
Use the sinusoidal reference generated in 1) as input to the system. Plot the tracking error, see the couplings also in X, Y directions. Try to change the stiffness in the X,Y directions and see that the tracking improves. See the joints are moving randomly because there is no control in the nullspace and we have 6 DoFs with a 3D task. sTry to add a damping term in the joint space.

\quad

\noindent
5) \textit{ End-effector position  PD control + Gravity Compensation:}
Add gravity compensation (in the operational space $G= J^{-T}g$). Give the step reference generated in 2) as input to the system. Check there is no longer a  tracking error at steady state is  but it still remains during the transient. 

\quad

\noindent
6) \textit{PD control  + Gravity Compensation + Feed-Forward term:}
Add also feed-forward term to the PD control (advice: use the reflected inertia at the end-efffector). Give the sinusoidal reference generated in 1). See that the tracking error is improved. 

\quad

\noindent
7) \textit{Operational space inverse dynamics:}
Implement operational space inverse dynamics with a postural task in the null space to attract the robot configuration q0 (advice tau0 = Kpostural*(q0 -q)).  Observe that the tracking is improved, there are no coupling between different directions anymore.  Observe that if you reduce the gains the tracking remains good apart in the discontinuity of the trajectory.  

\quad

\noindent
8) \textit{Operational space inverse dynamics + h :}
rather than compensate for u (bias term at the end effector  compensate for h (bias terms at joint level) and observe the result is the same.

\quad

\noindent
9) \textit{External Force:}
With constant reference $x_0$,  give $F_{ext}$ on $Z$ direction and check that it is affecting only the end-effector direction Z and not X and Y. If you reduce by half  the $K_x$ gains you should see the deflection due to $F_{ext}$ it doubles.

\quad

\noindent
10) \textit{Implement impedance control for poition and end-effector orientation :}
Add orientation  control of the end-effector (advice: the Jacobian is 6x6 now and there is no longer need of postural task). Implement the computation of the orientation error both with angle-axis and quaternions representations and compare they give the same result.
Note that you still have tracking errors because of gravity and inertial couplings.
Why the motion of the joint seems weird? This is what you get with a 6DoFs robot! There is no null-space left. Is necessary a 7 DoFs robot, if you want a more "natural" motion! Try to set a sinusoidal reference and plot the tracking.
 
 \quad
 
 \noindent
11) \textit{Implement full 6D inverse-dynamics  for position and end-effector orientation:}
Implement full inverse dynamics to remove inertial effects and gravity.
Observe that the $\Lambda$ computed with the 6x6 Jacobian can become singular. How do you solve this?
(hint: set a threshold for the singular values in the SVD decomposition in the np.linalg.pinv functions). 
 
\end{document}