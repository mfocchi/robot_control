
\documentclass{report}   

\usepackage{graphicx}
\usepackage[labelformat=empty]{caption}
\usepackage{subcaption}

\usepackage{multirow}
\usepackage{graphics,graphicx} % for pdf, bitmapped graphics files
\usepackage{epsfig} % for postscript graphics files
%\usepackage{mathptmx} % assume new font selection scheme installed
\usepackage{times} % assumes new font selection scheme installed
%\usepackage{amssymb}  % assumes amsmath package installed

\usepackage{amsmath} % assumes amsmath package installed
\usepackage{amsfonts}
%\usepackage{amssymb}  % assumes amsmath package installed
\usepackage{url}

\newcommand{\Rnum}{\mathbb{R}} % Symbol fo the real numbers set
\newcommand{\mat}[1]{\ensuremath{\begin{bmatrix}#1\end{bmatrix}}}	% matrix
\newcommand{\myparagraph}[1]{\paragraph{#1}\mbox{}\\}

\begin{document}


\section*{LAB 3: Contact consistent fixed base dynamics}


\quad

\noindent
1) \textit{Simulation of contact consistent dynamics:}
Generate a sinusoidal reference for Shoulder Lift joint with amplitude 0.6 rad and frequency 1 Hz.
Implement a constraint consistent dynamics where a point (3D) contact is possible only at the end-effector (ee\_link). 
Consider the appropriate projection of the dynamics before integrating the accelerations. 
Use the pre-implemented PD controller with gravity compensation and the logic 
to deal with the contact. What is the best way to compute the term $\dot{J}\dot{q}$? (hint: since $\ddot{x} = J\ddot{q} + \dot{J}\dot{q}$ compute the acceleration at the end-effector while setting $\ddot{q}= 0$).
Hint: consider the instantaneous correction of the joint velocity at the occurrence of impact.
Verify that the linear part of the twist at the end-effector equals $J\dot{q}$.

\quad

\noindent
2) \textit{Contact forces disappear when projecting the dynamics:}
Check contact force disappeared when projected after the projection in the null-space of $J^T$  (i.e. $ N_c^TJ^Tf = 0$) 
What happens if you use the Moore-penrose pseudo-inverse to compute the projector?
Plot also the torques in the row-space of $J^T$, check they are barely zero 
during the contact because there is almost no internal joint motion. 
Compare them with a plot of the joint torques.
The motion during the contact depends mainly only on the null-space projector $N_c^T$
that "cuts-out" the torques that generate contact forces, leaving only the ones that generate internal motions (in this case very small).

\quad

\noindent
3) \textit{Constraint consistent joint reference:}
Try to double the amplitude of the reference of Shoulder List joint to 1.2 rad.
Design a reference trajectory that is consistent with the contact (hint: compute $\dot{q}^{d}$ project with $N_c$ and integrate to get $q^d$.
Are there some internal motions now? are the torque in the row space different than zero?  

\quad

\noindent
4) \textit{Gauss principle of least effort}:
Verify that the Gauss principle of least constraint is satisfied (e.g. solve the QP where you minimize the distance w.r.t the unconstraint accelerations under the contact constraint).

\quad

\noindent
5)  \textit{Change in the contact location}:
Modify the code in order to allow the contact at a different location (e.g. origin of wrist\_3\_link frame).  Verify that the end-effector penetrates the ground. 

\quad

\noindent
6) \textit{Check the shifting law (Optional):}
The twist at ee\_link is $v_{e}$, twist at origin of wrist\_3\_joint is: $v_o$. Since they belong to the same rigid body (wrist\_3\_link) they are linked the shifting law, through by time-invariant \textit{motion} transform ${}_{e}X_o \in \Rnum^{6 \times 6}$:

\begin{align}
  v_{e} = {}_{ee}X_o v_o\\
  J_{e} \dot{q} = {}_{ee}X_o J_o \dot{q}
	\label{fig:}
\end{align}

therefore, also for the jacobians holds:

\begin{align}
J_{e}  = {}_{e}X_o J_o
\label{fig:}
\end{align}

where: 
\begin{align}
{}_{e}X_o  = \mat{ {}_{o}R_e   &  -{}_o R_e [{}_{e}t]_{\times} \\
					0             &     {}_oR_{e}} ^{-1}
\label{fig:}
\end{align}

Where ${}_{e}t$ is the relative position of the origin of frame ${o}$ w.r.t. frame $e$ expressed 
in frame ${e}$.
By activating the TF function in \textit{rviz} check the relative location of the frames. 

 
\end{document}